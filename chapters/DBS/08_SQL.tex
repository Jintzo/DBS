\section{Relationale Datenbanksprachen}
\label{sec:sql}

\textbf{Aggregatfunktionen}
\begin{items}
	\item Prinzip: Berechnung eines Werts aus Werten eines Attributs
	\item \underline{Join (natural)}: Kartesisches Produkt zweier Relationen
	\item Weitere in Standard SQL: count(), sum(), min(), max(), avg()
\end{items}

\textbf{SQL-Kern}
\begin{items}
	\item \underline{select} \\*
		Projektionsliste, \\*
		arithmetische Operationen und Aggregatfunktionen
	\item \underline{from} \\*
		zu verwendende Relationen, ggf. Umbenennungen
	\item \underline{where} \\*
		Selektions- und Verbundbedingungen \\*
		geschachtelte Anfragen (wieder SFW-Block)
	\item \underline{group by} \\*
		Gruppierung für Aggregatfunktionen
	\item \underline{having} \\*
		Selektionsbedingungen an Gruppen
\end{items}