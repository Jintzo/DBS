\section{Cloudsysteme -- Konsistenz}
\label{sec:cloudkonsistenz}

\textbf{Verteilung}
\begin{items}
	\item \underline{Vorteile} (scheinbar):
	\begin{enumeration}
		\item Leselastverteilung
		\item Beschleunigung (durch höhere Lokalität)
		\item Höhere Ausfallsicherheit
	\end{enumeration}
	\item \underline{Nachteile}:
	\begin{enumeration}
		\item Transaktionen müssen auf Knoten gleich angeordnet sein
		\item Widerspruchsfreie Anordnungsentscheidungen nötig für Konfliktfreiheit \( \leadsto \) schlechte Skalierbarkeit
		\item Für Konsistenz müssen alle Knoten verfügbar sein \( \leadsto \) geringere Ausfallsicherheit
	\end{enumeration}
	\item \( \leadsto \) Netzwerkpartitionierung
	\item \underline{CAP-Theorem}: Wenn Netzwerkpartitionierung möglich, dann sind hohe Verfügbarkeit und Datenbestandskonsistenz unvereinbar
\end{items}

\textbf{Eventual Consistency}
\begin{items}
	\item ``Wenn ab Zeitpunkt keine Änderungen mehr, dann werden irgendwann alle Lesezugriffe gleichen Wert zurückliefern''
	\item Alternativ: ``\dots dann werden irgendwann alle Lesezugriffe zuletzt geschriebenen Wert zurückliefern''
	\item Beispiel (social network): Netzwerkpartition
	\begin{items}
		\item Starke Konsistenz: Vorübergehend keine Postings möglich
		\item Eventual Consistency: User kann Posting schreiben, Follower sehen es sobald möglich
	\end{items}
\end{items}

\begin{fragen}
	\begin{enumeration}
		\item Geben Sie die Probleme mit dem klassischen, starken Konsistenzbegriff im verteilten Fall wieder.
		\item Bekommt man mit \emph{eventual consitency} irgendeine Form von Sicherheit? Begründen Sie Ihre Antwort.
		\item Warum kann man im Bank-Kontext in manchen Situationen doch auf starke, klassische Konsistenz verzichten?
		\item Geben Sie ein weiteres Beispiel für eine Folge von Operationen, deren Anordnung egal ist.
	\end{enumeration}
\end{fragen}