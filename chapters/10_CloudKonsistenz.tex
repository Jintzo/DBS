\section{Cloudsysteme --- Konsistenz}
\label{sec:cloudkonsistenz}

\paragraph{Verteilung}
\begin{itemize}
	\item \textbf{Vorteile} (scheinbar): \\*
		+ Leselastverteilung \\*
		+ Beschleunigung (durch höhere Lokalität) \\*
		+ Höhere Ausfallsicherheit
	\item \textbf{Nachteile}: \\*
		- Transaktionen müssen auf Knoten gleich angeordnet sein \\*
		- Widerspruchsfreie Anordnungsentscheidungen nötig für Konfliktfreiheit \( \leadsto \) schlechte Skalierbarkeit \\*
		- Für Konsistenz müssen alle Knoten verfügbar sein \( \leadsto \) geringere Ausfallsicherheit
	\item \( \leadsto \) Netzwerkpartitionierung
	\item \textbf{CAP-Theorem}: Wenn Netzwerkpartitionierung möglich, dann sind hohe Verfügbarkeit und Datenbestandskonsistenz unvereinbar
\end{itemize}

\paragraph{Eventual Consistency}
\begin{itemize}
	\item ``Wenn ab Zeitpunkt keine Änderungen mehr, dann werden irgendwann alle Lesezugriffe gleichen Wert zurückliefern''
	\item Alternativ: ``\dots dann werden irgendwann alle Lesezugriffe zuletzt geschriebenen Wert zurückliefern''
	\item Beispiel (social network): Netzwerkpartition
	\begin{itemize}
		\item Starke Konsistenz: Vorübergehend keine Postings möglich
		\item Eventual Consistency: User kann Posting schreiben, Follower sehen es sobald möglich
	\end{itemize}
\end{itemize}

\begin{fragen}
	\item Geben Sie die Probleme mit dem klassischen, starken Konsistenzbegriff im verteilten Fall wieder.
	\item Bekommt man mit \emph{eventual consitency} irgendeine Form von Sicherheit? Begründen Sie Ihre Antwort.
	\item Warum kann man im Bank-Kontext in manchen Situationen doch auf starke, klassische Konsistenz verzichten?
	\item Geben Sie ein weiteres Beispiel für eine Folge von Operationen, deren Anordnung egal ist.
\end{fragen}